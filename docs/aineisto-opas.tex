\documentclass[]{book}
\usepackage{lmodern}
\usepackage{amssymb,amsmath}
\usepackage{ifxetex,ifluatex}
\usepackage{fixltx2e} % provides \textsubscript
\ifnum 0\ifxetex 1\fi\ifluatex 1\fi=0 % if pdftex
  \usepackage[T1]{fontenc}
  \usepackage[utf8]{inputenc}
\else % if luatex or xelatex
  \ifxetex
    \usepackage{mathspec}
  \else
    \usepackage{fontspec}
  \fi
  \defaultfontfeatures{Ligatures=TeX,Scale=MatchLowercase}
\fi
% use upquote if available, for straight quotes in verbatim environments
\IfFileExists{upquote.sty}{\usepackage{upquote}}{}
% use microtype if available
\IfFileExists{microtype.sty}{%
\usepackage{microtype}
\UseMicrotypeSet[protrusion]{basicmath} % disable protrusion for tt fonts
}{}
\usepackage[margin=1in]{geometry}
\usepackage{hyperref}
\hypersetup{unicode=true,
            pdftitle={Kielitieteellisen aineiston kerääjän opas},
            pdfauthor={Niko Partanen},
            pdfborder={0 0 0},
            breaklinks=true}
\urlstyle{same}  % don't use monospace font for urls
\usepackage{natbib}
\bibliographystyle{apalike}
\usepackage{longtable,booktabs}
\usepackage{graphicx,grffile}
\makeatletter
\def\maxwidth{\ifdim\Gin@nat@width>\linewidth\linewidth\else\Gin@nat@width\fi}
\def\maxheight{\ifdim\Gin@nat@height>\textheight\textheight\else\Gin@nat@height\fi}
\makeatother
% Scale images if necessary, so that they will not overflow the page
% margins by default, and it is still possible to overwrite the defaults
% using explicit options in \includegraphics[width, height, ...]{}
\setkeys{Gin}{width=\maxwidth,height=\maxheight,keepaspectratio}
\IfFileExists{parskip.sty}{%
\usepackage{parskip}
}{% else
\setlength{\parindent}{0pt}
\setlength{\parskip}{6pt plus 2pt minus 1pt}
}
\setlength{\emergencystretch}{3em}  % prevent overfull lines
\providecommand{\tightlist}{%
  \setlength{\itemsep}{0pt}\setlength{\parskip}{0pt}}
\setcounter{secnumdepth}{5}
% Redefines (sub)paragraphs to behave more like sections
\ifx\paragraph\undefined\else
\let\oldparagraph\paragraph
\renewcommand{\paragraph}[1]{\oldparagraph{#1}\mbox{}}
\fi
\ifx\subparagraph\undefined\else
\let\oldsubparagraph\subparagraph
\renewcommand{\subparagraph}[1]{\oldsubparagraph{#1}\mbox{}}
\fi

%%% Use protect on footnotes to avoid problems with footnotes in titles
\let\rmarkdownfootnote\footnote%
\def\footnote{\protect\rmarkdownfootnote}

%%% Change title format to be more compact
\usepackage{titling}

% Create subtitle command for use in maketitle
\newcommand{\subtitle}[1]{
  \posttitle{
    \begin{center}\large#1\end{center}
    }
}

\setlength{\droptitle}{-2em}

  \title{Kielitieteellisen aineiston kerääjän opas}
    \pretitle{\vspace{\droptitle}\centering\huge}
  \posttitle{\par}
    \author{Niko Partanen}
    \preauthor{\centering\large\emph}
  \postauthor{\par}
      \predate{\centering\large\emph}
  \postdate{\par}
    \date{2018-11-20}

\usepackage{booktabs}
\usepackage{amsthm}
\makeatletter
\def\thm@space@setup{%
  \thm@preskip=8pt plus 2pt minus 4pt
  \thm@postskip=\thm@preskip
}
\makeatother

\begin{document}
\maketitle

{
\setcounter{tocdepth}{1}
\tableofcontents
}
\hypertarget{esittely}{%
\chapter{Esittely}\label{esittely}}

Kielitieteellisiä aineistoja on kerätty jo satojen vuosien ajan, joten
myös uudet aineistot on syytä nähdä tällaisessa syvemmässä
historiallisessa kontekstissa. Suomalais-ugrilaisten kielten osalta voi
sanoa, että 1800-luvun lopulla kerätyt aineistot alkavat
transkriptioltaan ja kuvailutiedoiltaan muistuttaa 1900-luvun aikana
kerättyjä. Tätä vanhemmat aineistot ovat myös hyvin arvokkaita, mutta
niiden käyttöön liittyy eri tavalla tehtävä tulkinta. Selkeä ero on
esimerkiksi siinä, että vanhimmat aineistot eivät useinkaan sisällä
tietoja kielenoppaista. Kieltä ajateltiin ehkä eri tavalla puhujasta
irrallisena entiteettinä. Toisaalta 1800-luvun lopun aineistoissa on
usein yhtä paljon, tai vähän, kuvailutietoja kuin uudemmissakin
aineistoissa.

\hypertarget{kuka-mina-olen}{%
\section{Kuka minä olen?}\label{kuka-mina-olen}}

Työskentelen Kotimaisten kielten keskuksessa kirjasto- ja
aineistoyksikössä erityisasiantuntijana. Olen työskennellyt aiemmin mm.
Freiburgin yliopistossa, Hamburger Zentrum für Sprachkorpora
-keskuksessa sekä Pariisin ENS-yliopiston Lattice-laboratoriossa.
Pääasiallinen kieli jota tutkin on komisyrjääni. Olen myös kiinnostunut
itämerensuomalaisista kielistä ja permiläisistä kielistä yleisesti. Olen
käyttänyt paljon kieliteknologiaa ja jonkin verran
koneoppimismenetelmiä, mutta yleensä käyttäjän perspektiivistä, haluten
saada näillä jonkin vaikean tai työlään työvaiheen pois käsistäni.

Teen väitöskirjaani komisyrjäänin murteellisesta variaatiosta. Olen
julkaissut viime vuosina erilaisia artikkeleja, mutten ole varma,
tuleeko näistä mikään väitöskirjaani. En ole varsinaisesti tehnyt
asioita suositellun akateemisen urakehityksen mukaisesti. CV:ni on
luettavissa mm.
\href{https://github.com/nikopartanen/cv/blob/master/niko_partanen_cv_fin.pdf}{täällä}.

\hypertarget{muutamia-termeja-avattuina}{%
\subsection{Muutamia termejä
avattuina}\label{muutamia-termeja-avattuina}}

\begin{enumerate}
\def\labelenumi{\arabic{enumi}.}
\tightlist
\item
  Korpus: Jotenkin valittu aineisto, johon on usein tehty erilaisia
  annotointeja
\item
  Annotointi: Jonkin asian systemaattinen merkintä, esim. sanaluokka,
  lauseenjäsen, elollisuus yms.
\item
  Lisenssi: Ehdot, joilla aineistoa saa käyttää ja jakaa
\end{enumerate}

\hypertarget{tutkimusetiikka}{%
\chapter{Tutkimusetiikka}\label{tutkimusetiikka}}

Tänne jotain.

\hypertarget{aineistojen-kaytto}{%
\section{Aineistojen käyttö}\label{aineistojen-kaytto}}

Periaatteessa mitä tahansa tietyllä kielellä olevaa materiaalia voi
käyttää tutkimusaineistona. Toisaalta aiheen tarkempi miettiminen on
hyvin kannattavaa. Usein haluat esimerkkejä jostain tietystä ilmiöstä
tai asiasta. Lopullisessa analyysissasi voi olla hyvin tärkeää,
millaisista konteksteista esimerkit tulevat. Tekstilaji, aika,
käännöksellisyys ja muut seikat vaikuttavat väistämättä. Voit tietysti
lähteä liikkeelle siitä, että keräät jonkin ilmiön kaikki esimerkit,
jotka vain voit löytää, mutta tämä on käytännössä mahdoton tavoite.
Tuloksena on, että olet saanut tietystä rajallisesta aineistosta ne
esimerkit, jotka olet sattunut löytämään.

Näin ollen tutkimasi asian esiintymisfrekvenssi vaikuttaa suoraan
siihen, kuinka paljon aineistoa tarvitset.

Annotoitujen korpusten hyvä puoli on siinä, että niiden sisältö on
useimmiten valittu jonkin prinsiipin mukaan. Ihannetilanteessa tämä
periaate on kuvattu jossain ja pystytään myöhemminkin perustelemaan
järkevästi. Huomaa, että aina näin ei ole. Erityisesti kielten
dokumentaatioprojekteissa tehdyt korpukset ovat usein ns.
opportunistisia, ja tavoitteena on ollut kerätä kaikki mahdollinen. Se,
mitä on lopulta litteroitu ja annotoitu, on kuitenkin useimmiten ollut
hyvin tietoinen prosessi.

Myös annotaatiot on aina tehty jonkin tietyn periaatteen mukaan. Tavoite
on aina, että samaa annotaatiomallia käytetään systemaattisesti koko
aineistossa. Tässä tavoitteessa harvoin onnistutaan, ja jokaiseen
korpukseen jää jonkinlaisia virheitä. Älä luota yhdenkään aineiston
olevan täydellisen systemaattinen. Jos jokin asia ei ole kunnossa, voit
myös aina korjata sen itse. Voitko jakaa korjauksesi muiden kanssa
riippuu mm. lisensseistä.

\hypertarget{lisenssit-ja-tekijanoikeudet}{%
\section{Lisenssit ja
tekijänoikeudet}\label{lisenssit-ja-tekijanoikeudet}}

Kaikkea aineistoa ei ole lisensoitu. Lisenssien puuttumattomuus on
erityisen yleistä tietyillä kulttuurialueilla, esimerkiksi Venäjällä.
Lisenssi käytännössä tarkoittaa sitä, että aineiston tekijä antaa muille
oikeuden tehdä aineistolla tiettyjä asioita.

Lisensoimatonta materiaalia ei periaatteessa voi käyttää kuin tietyin
rajallisin ehdoin.

Avoimet lisenssit ovat erityisen pitkällä ohjelmistokehityksen puolella.

Kielitieteellisiin aineistoihin käytetään usein lisenssiä CC-BY. Tämä
tarkoittaa, että aineiston käyttäjän on viitattava alkuperäiseen
aineistoon ja tekijään. Aineistoista voi tehdä myös uusia versioita,
mutta niidenkin yhteydessä on aina oltava viittaus.

Viittaus on pakko tehdä tieteellisessä käytössä myös hyvän tieteellisen
käytännön mukaisesti. Ei ole selvää, kuinka tiukasti CC-lisenssien
rikkomisia valvotaan, ja käytännössä lisenssi on aina peli siitä,
kenellä on eniten rahaa. Suuret yritykset rikkovat tekijänoikeuksia
säännöllisesti sen turvin, ettei loukatuilla tahoilla ole resursseja
haastaa heitä oikeuteen. Toisaalta hyvän tieteellisen käytännön
rikkomisesta on melko selviä toimintamalleja ja rangaistuksia.

\hypertarget{lisenssityypit}{%
\subsection{Lisenssityypit}\label{lisenssityypit}}

\begin{itemize}
\tightlist
\item
  CC-BY tarkoittaa, että lähteeseen on viitattava
\item
  CC-BY-NC kieltää kaupallisen käytön

  \begin{itemize}
  \tightlist
  \item
    On erittäin vaikeaa määritellä, mitä on kaupallinen käyttö
  \end{itemize}
\item
  CC-BY-ND kieltää johdannaisteokset (no derivations)

  \begin{itemize}
  \tightlist
  \item
    Tiukasti tulkittuna mikä tahansa jatkokäyttö menee tämän allen
  \end{itemize}
\item
  ACA ei varsinaisesti ole lisenssi, vaan käyttöehto, että aineistoa saa
  käyttää tutkimustyössä. Sitä ei siis saa jakaa edelleen, ja
  käytännössä kaikki käyttörajoitukset ovat voimassa.
\item
  CC-0 / Public Domain ei ole lisenssi, vaan oikeastaan lisenssin
  puuttuminen
\end{itemize}

CC-0 on ajoittain esitetty suositelluksi lisenssiksi. On hieman
epäselvää, salliiko laki kuitenkaan tekijänoikeuksista luopumista.
Toisaalta ei ole selvää, tajuavatko tutkijat, että jo tieteellinen
käytäntö pakottaa viittaamaan.

\hypertarget{public-domain}{%
\subsection{Public Domain}\label{public-domain}}

On tiettyjä tapauksia, jolloin materiaali muuttuu tekijänoikeusvapaaksi.
Selkein tapaus on ikä. 70 vuotta tekijän kuolemasta on Suomessa
vallitseva raja to-menetykselle. Esimerkiksi Venäjällä tilanne voi olla
erilainen, sillä esimerkiksi kansallissankareille ja sodan vainojen
uhreille on erilaisia säädöksiä tämän suhteen.

Tekijänoikeus voi myös raueta, jos kyseessä on orpoteos. Tämä
tarkoittaa, että tekijällä ei ole perillisiä, joilla olisi oikeus
teokseen. Tällöin hyvinkin tuore teos voisi periaatteessa olla
lisenssivapaa. Käytännössä tämän todistaminen Suomessa on ilmeisen
hankalaa, mutta esimerkiksi Venäjän kirjastot ovat tietyissä tapauksissa
tehneet tätä.

Tietyn teoksen olettaminen tekijänoikeusvapaaksi on aina tietty riski.
Turvallisinta se on silloin, kun jokin kansallisesti virallinen taho on
jo tehnyt määritelmän puolestasi. Tällöinkin viittaaminen juuri tuohon
versioon on käyttäjän kannalta selkeintä ja turvallisinta.

\hypertarget{tutkimusaineistojen-koonti-ja-lisenssi}{%
\subsection{Tutkimusaineistojen koonti ja
lisenssi}\label{tutkimusaineistojen-koonti-ja-lisenssi}}

Kun keräät tutkimukseesi materiaalia eri lähteistä, teet käytännössä
uutta aineistoa. Sinun on otettava käyttämiesi materiaalien lisenssit,
jotta voit tietää, saako aineistoasi jakaa eteenpäin ja millaisin
ehdoin. Nykyisin on modernia ja toivottavaa esimerkiksi jakaa
tutkimusaineistosi artikkelisi tai gradusi liitteenä. Tai voit
rekisteröidä käyttämäsi aineiston esimerkiksi Zenodo-palvelussa, ja
viitata siihen erikseen. Näin jokainen julkaisu olisi käytännössä
eräänlainen kaksoisjulkaisu, jossa julkaisisit erillisesti tutkimuksesi
ja siinä käyttämäsi aineiston.

Jos tästä halutaan tehdä oikein fiiniä, olisi syytä julkaista myös
erillinen teksti, jossa kuvaat käyttämäsi aineiston. Suuremman
tutkimuksen kuten gradusi yhteydessä tuo voisi hyvin olla jollain
tavalla osa metodilukua.

\hypertarget{annotaatioiden-valttamattomyys-ja-riittamattomyys}{%
\section{Annotaatioiden välttämättömyys ja
riittämättömyys}\label{annotaatioiden-valttamattomyys-ja-riittamattomyys}}

Kuten totesin, annotoidut aineistot ovat usein hyvin hyödyllisiä.
Tehokas hakeminen niistä mahdollistaa monessa kohtaa nopeamman halutun
otoksen saamisen, jopa sen täydellisen automatisoinnin. Tämä ei
kuitenkaan tarkoita, että työ olisi heti tehty. Tutkimuskysymyksestä
riippuen esimerkkien syvällisempi annotointi on usein välttämätöntä.
Tämä sinun on tehtävä itse.

On myös hyvin mahdollista, että olet eri mieltä korpuksen annotointien
kanssa. Mieti tällöin, koodaavatko annotaatiot saman tiedon, kuin mitä
itsekin merkitsisit, vai onko ongelma siinä, että jotain distinktiota
oikeasti ei merkitä. Jos et pidä valitusta mallista, on se mahdollista
aina muuttaa johonkin muuhun formaattiin, mutta puuttuvaa tietoa ei saa
sinne kuin itse lisäämällä.

Hyvin normaali tilanne on, että kun teet jonkinlaisen haun korpukseen,
sisältää se esimerkiksi 95\% juuri sitä mitä haluat. Joukossa on joka
tapauksessa lähes aina jotain turhaakin, esimerkiksi annotaatiovirheitä
tai johonkin muuhun liittyviä esimerkkejä. Joka tapauksessa näiden
käyminen läpi jatkoannotaatiotyön yhteydessä on verrattain nopeaa. Mieti
myös tarkkaan sitä, onko korpuksessa nykyisen haun ulkopuolelle jääviä
esimerkkejä, jotka kuitenkin tarvitsisit.

Yksi hyvä formaatti aineiston käsittelylle on perinteinen
taulukkolaskentaohjelma, kuten LibreOffice tai Excel. Excelin kanssa
kannattaa olla tarkkana, että se ei sotke fontteja. Yksi hyvä vaihtoehto
on niin sanottu CSV-tiedosto. Tämä tarkoittaa comma-separated-values.
Pilkun sijasta erottava merkki voi myös olla sarkain tai puolipiste.
Tällaisen voi myös avata Excelissä ja muualla, ja se on monissa
tapauksissa hyvin elegantti vaihtoehto.

\hypertarget{aineiston-ja-tutkimuksen-erillisyys-ja-yhteys}{%
\section{Aineiston ja tutkimuksen erillisyys ja
yhteys}\label{aineiston-ja-tutkimuksen-erillisyys-ja-yhteys}}

\hypertarget{versiokontrolli}{%
\section{Versiokontrolli}\label{versiokontrolli}}

\hypertarget{erilaiset-aineistot}{%
\chapter{Erilaiset aineistot}\label{erilaiset-aineistot}}

Mielestäni tutkijan kannattaisi pyrkiä siihen, että hänen käyttämänsä
aineisto olisi digitaalisessa formaatissa. Tietysti on mahdollista etsiä
esimerkkejä myös käymällä kirjoja läpi omin silmin, mutta nykyään on
ehkä usein muitakin vaihtoehtoja.

Tekstitiedostot käyvät sinällään jo moneen. On syytä tutustua
säännöllisiin lausekkeisiin, tai ainakin erilaisiin hauissa toimiviin
wild card -merkkeihin. Ohjelmat kuten AntConc ovat kokeilemisen
arvoisia.

Jos haluat käyttää aineistona jotain kirjaa tai muuta teosta, joka
sinulla on esimerkiksi skannattuna, saa tekstin siitä nykyään helposti
ulos OCR-ohjelmilla. Kerron mielelläni lisää!

Iso ero on myös puhuttujen ja kirjoitettujen aineistojen välillä.
Puhutut aineistot tulevat usein formaateissa, jotka mahdollistavat
lisäannotointien teon. Esimerkiksi ELAN ja Exmaralda-ohjelmien tuottamat
tiedostot ovat sellaisia, että niissä on omat kerrokset erilaisilla
annotointityypeille. Voit siis tehdä uuden kerroksen, jossa on
annotoituna vain sinun tutkimuksellesi tarpeelliset asiat niissä
esimerkeissä, joita haluat käyttää.

\hypertarget{aineistoesimerkkeja}{%
\section{Aineistoesimerkkejä}\label{aineistoesimerkkeja}}

\begin{itemize}
\tightlist
\item
  \href{https://www.kielipankki.fi/}{Kielipankki}
\item
  \href{https://www.hf.uio.no/iln/english/about/organization/text-laboratory/services/index.html\#speech}{Oslon
  yliopiston korpukset}

  \begin{itemize}
  \tightlist
  \item
    Esimerkiksi
    \href{http://tekstlab.uio.no/glossa/html/index_dev.php?corpus=ruija}{Ruija-korpus}
  \item
    \href{https://tekstlab.uio.no/glossa2/saami}{LIA sápmi} hyvä
    esimerkki akateemisesta lisenssistä
  \end{itemize}
\item
  \href{https://corpora.uni-hamburg.de/hzsk/}{HZSK}

  \begin{itemize}
  \tightlist
  \item
    Selkuppi, nganasaani
  \end{itemize}
\item
  \href{http://universaldependencies.org/}{Universal Dependencies}
\item
  \href{http://www.babel.gwi.uni-muenchen.de/}{Ob-Babel}
\item
  Oulun saameaineistot
\end{itemize}

\hypertarget{venajalla-tyypillisia-aineistoja}{%
\subsection{Venäjällä tyypillisiä
aineistoja}\label{venajalla-tyypillisia-aineistoja}}

\begin{itemize}
\item
  \href{http://komicorpora.ru}{Komin kansallinen korpus}
\item
  \href{http://web-corpora.net/UdmurtCorpus/search/}{Udmurt corpus}
\item
  \href{http://beserman.ru/corpus/search/}{Beserman corpus}
\item
  \href{http://selkup.org/}{Selkup.org}
\item
  \textbf{Ongelma usein puutteellinen export-toiminto, mutta asiat ehkä
  kehittyvät.}
\end{itemize}

\hypertarget{satunnaista}{%
\subsection{Satunnaista}\label{satunnaista}}

\begin{itemize}
\tightlist
\item
  \href{http://www.helsinki.fi/~tasalmin/tn_corpus.html}{Tundra Nenets
  sample sentence corpus}
\item
  \href{http://www.iling-ran.ru/gusev/Nganasan/}{Valentin Gusevin
  nganasaani-korpus}
\end{itemize}

\hypertarget{tutkimusetiikka-1}{%
\chapter{Tutkimusetiikka}\label{tutkimusetiikka-1}}

\begin{quote}
Tieteellisen tutkimuksen luotettavuus ja tulosten uskottavuus
edellyttävät, että tutkimuksessa noudatetaan hyvää tieteellistä
käytäntöä. Vastuu hyvän tieteellisen käytännön noudattamisesta kuuluu
koko tiedeyhteisölle ja jokaiselle tutkijalle. Helsingin yliopisto on
sitoutunut Tutkimuseettisen neuvottelukunnan ohjeisiin hyvästä
tieteellisestä käytännöstä ja sen loukkausten käsittelemisestä.
\href{https://www.helsinki.fi/fi/tutkimus/tutkimusymparisto/tutkimusetiikka}{HY}
\end{quote}

Kielitieteelliset aineistot poikkeuksetta käsittelevät ihmisiä. Joku on
tuottanut tai kirjoittanut lauseet, joita tutkimme.

\hypertarget{tietosuojalain-nykytila}{%
\section{Tietosuojalain nykytila}\label{tietosuojalain-nykytila}}

\begin{quote}
Eduskunnan on tarkoitus hyväksyä asetusta täydentävä tietosuojalaki
syksyllä 2018. Asetuksen edellyttämät kansalliset lainsäädäntömuutokset
ja yksityiskohdat ovat vielä auki. Tietosuoja-asetusta on tästä
huolimatta sovellettava 25.5.2018 alkaen.
\href{https://ek.fi/mita-teemme/yrityslainsaadanto/tietosuojalainsaadanto/tietopaketti-yrityksille-on-aika-valmistautua-eun-yleiseen-tietosuoja-asetukseen/}{(lähde)}
\end{quote}

\begin{quote}
Kansallista liikkumavaraa yleisen tietosuoja-asetuksen täydentämiseksi
ehdotetaan käytettävän tilanteissa, joissa henkilötietolain kumoamisesta
seuraisi tarve kansalliseen sääntelyyn, esimerkiksi henkilötietojen
käsittelyn oikeusperusteen säilyttämiseksi eräissä tilanteissa tai
tieteellisen tutkimuksen edellytysten säilyttämiseksi mahdollisimman
pitkälle nykyisen kaltaisina.
\href{https://www.eduskunta.fi/FI/vaski/HallituksenEsitys/Sivut/HE_9+2018.aspx}{(lähde)}
\end{quote}

\begin{quote}
Edellytyksenä on niin ikään, että henkilörekisteriä käytetään ja siitä
luovutetaan henkilötietoja vain historiallista tai tieteellistä
tutkimusta varten sekä muutoinkin toimitaan niin, että tiettyä henkilöä
koskevat tiedot eivät paljastu ulkopuolisille, ja että henkilörekisteri
hävitetään tai siirretään arkistoitavaksi tai sen tiedot muutetaan
sellaiseen muotoon, ettei tiedon kohde ole niistä tunnistettavissa, kun
henkilötiedot eivät enää ole tarpeen tutkimuksen suorittamiseksi tai sen
tulosten asianmukaisuuden varmistamiseksi.
\href{https://www.eduskunta.fi/FI/vaski/HallituksenEsitys/Sivut/HE_9+2018.aspx}{(lähde)}
\end{quote}

\hypertarget{gdpr}{%
\section{GDPR}\label{gdpr}}

Kesällä 2018 EU:n yleinen tietosuoja-asetus (General Data Protection
Regulation, GDPR) astui voimaan. Se antaa yksityishenkilöille paremman
suojan heidän henkilötiedoilleen ja keinot hallita niiden käsittelyä.

Tutkija, joka kerää tutkimusaineistoa, käsittelee lähes poikkeuksetta
henkilötietoja. On mahdollista kerätä tietoja esimerkiksi anonyymin
lomakkeen kautta, mutta tutkimuksen itsensä kannalta on yleensä tärkeä
tietää kuka puhuja on, mikä on hänen sukupuolensa, ikänsä,
syntymäpaikkansa tai äidinkielensä. Tarkka kerättävä tieto riippuu
tutkimuskysymyksestä.

Henkilötietoja ovat esimerkiksi {[}(lähde:
Tietosuoja.fi)(\url{https://tietosuoja.fi/gdpr}){]}:

\begin{itemize}
\tightlist
\item
  nimi
\item
  kotiosoite
\item
  sähköpostiosoite, kuten
  \href{mailto:etunimi.sukunimi@yritys.com}{\nolinkurl{etunimi.sukunimi@yritys.com}}
\item
  puhelinnumero
\item
  henkilökortin numero
\item
  auton rekisterinumero
\item
  paikannustiedot
\item
  IP-osoite
\item
  potilastiedot
\item
  isovanhempien perinnöllisiä sairauksia koskevat tiedot
\end{itemize}

Näiden ohella on tässä yhteydessä mainittava, että myös \textbf{ääni} on
henkilötieto. Tämä johtuu yksiselitteisesti siitä, että ihmiset pystyvät
erittäin tarkasti tunnistamaan kenen tahansa tutun henkilön äänen.
Toisaalta puheen automaattisen tunnistamisen, johon liittyy puhujan
tunnistus, tekniikat kehittyvät niin nopeasti, että jos äänitiedosto on
kuunneltavissa, on tämä käytännössä sama kuin että ainakin nimi olisi
myös saatavilla. Tämä täytyy ottaa huomioon.

Henkilötietojen käsittelyyn tarvitaan peruste. Tällaisia ovat
esimerkiksi:

\begin{itemize}
\tightlist
\item
  rekisteröidyn suostumus
\item
  sopimus
\item
  rekisterinpitäjän lakisääteinen velvoite
\item
  elintärkeiden etujen suojaaminen
\item
  yleinen etu ja julkinen valta
\item
  rekisterinpitäjän tai kolmannen osapuolen oikeutettu etu.
\end{itemize}

Ongelma suostumuksen ja sopimuksen käytössä käsittelyperiaatteena on se,
että ne mahdollistavat käytön vain tiettyyn tai useampaan sopimuksessa
määriteltyyn tarkoitukseen. Ei ole laillisesti mahdollista sopia, että
kerättyä aineistoa käytetään yleisesti kielitieteellisessä tutkimuksessa
tulevaisuudessa.

Perinteisesti kielitieteellisen aineistonkeruun yksi päätavoite on
kuitenkin ollut tallentaa näytteitä katoavista ja uhanalaistuvista
kielimuodoista tulevaa käyttöä varten. Tämä vastaa hengeltään hyvin
yleisen edun määrittelyä laissa:

\begin{quote}
Yleisen edun mukaista käsittelyä voi esimerkiksi olla henkilötietojen
käsittely tieteellisen tai historiallisen tutkimuksen tai tilastoinnin
tarkoituksia varten.
\end{quote}

Näin ollen akateemiseen instituution affilioituneen tutkijan tai
projektin työ melko yksiselitteisesti menisi tämän alle.

Aineistojen pitkäaikaissäilytys ja muut arkistointikysymykset selkenevät
vasta kun Suomen uusi arkistolaki valmistuu. Käytännössä kuitenkin
normaali tilanne olisi, että tutkijat voisivat tutkimusprojektiensa
yhteydessä arkistoida aineistot esimerkiksi Kotimaisten kielten
keskukseen ja Kielipankkiin, mistä he itsekin saisivat ne uudelleen
mahdollisia uusia tutkimuksiaan varten.

Käytännössä tapana on ollut, että tutkijat keräävät aineistoa ja
säilyttävät sitä omilla tietokoneillaan vuosikymmeniä. Käytännössä on
kuitenkin lain ja etiikan kannalta täysin mahdotonta, että tutkijat,
yksityishenkilöinä, säilyttävät henkilötietoja sisältävää aineistoa
tällä tavalla.

\hypertarget{data-management-plan}{%
\section{Data management plan}\label{data-management-plan}}

Tietojen käsittelystä on tehtävä suunnitelma seuraavissa tapauksissa:

\begin{itemize}
\tightlist
\item
  henkilötietojen käsittely organisaatiossa ei ole satunnaista
\item
  organisaation vastuulla oleva henkilötietojen käsittely
  todennäköisesti aiheuttaa riskin rekisteröidyn oikeuksille ja
  vapauksille tai
\item
  organisaatiossa käsitellään arkaluonteisia tietoja.
\end{itemize}

Nähdäkseni tutkijoiden tekemä aineiston käsittely väistämättä menee
kategoriaan, jossa henkilötietojen käsittely ei ole satunnaista. Näin
ollen jonkinlainen suunnitelma olisi syytä tehdä. Käytännössä tällaisia
vaaditaan nykyään enenevissä määrin myös projektihakemusten yhteydessä,
joten tässä mielessä luvassa ei ole mitään uutta.

Kyseessä on organisaation sisäinen asiakirja, jonka perusteella
tietojenkäsittelytoimien lainmukaisuutta voidaan arvioida. Nähdäkseni
siinä olisi syytä kuvata, kuka aineistoa käsittelee, missä ympäristössä,
minne sitä kopioidaan, ja kuinka aineisto järjestetään.

\hypertarget{erityiset-henkilotiedot}{%
\section{Erityiset henkilötiedot}\label{erityiset-henkilotiedot}}

Erityiset henkilötiedot koskettavat tutkimus siinä mielessä hyvin
voimakkaasti, että \textbf{niiden käsittely on lähtökohtaisesti
kiellettyä}. Erityisiksi henkilötiedoiksi lasketaan seuraavat:

\begin{itemize}
\tightlist
\item
  rotu tai etninen alkuperä
\item
  poliittisia mielipiteitä
\item
  uskonnollinen tai filosofinen vakaumus
\item
  ammattiliiton jäsenyys
\item
  terveyttä koskevia tietoja
\item
  seksuaalinen suuntautuminen tai käyttäytyminen
\item
  geneettisiä ja biometrisia tietoja henkilön tunnistamista varten
\end{itemize}

Jos voit tehdä tutkimuksesi keräämättä erityisiä henkilötietoja, niin
toimi näin. Tällä hetkellä, ymmärtääkseni, monikaan arkisto ei suostu
ottamaan tällaista aineistoa vastaan, eikä sitä saa käsitellä
esimerkiksi CSC:n IDA-ympäristössä.

Joka tapauksessa tämäntapainen käsittely olisi silti ehkä mahdollista
\href{https://tietosuoja.fi/erityisten-henkilotietoryhmien-kasittely}{(lähde:
Tietosuoja.fi)}:

\begin{quote}
Kun käsittely on tarpeen \textbf{yleisen edun mukaista arkistointia,
tieteellistä ja historiallista tutkimusta tai tilastointia varten}
tietosuoja-asetuksen mukaisesti unionin oikeuden tai jäsenvaltion
lainsäädännön nojalla. Käsittelyn mahdollistavan sääntelyn tulee olla
oikeassa suhteessa käsittelyn tavoitteeseen nähden, ja siinä tulee
noudattaa keskeisiltä osin oikeutta henkilötietojen suojaan. Tässä
yhteydessä on myös säädettävä toimenpiteistä, joilla suojataan
rekisteröidyn perusoikeudet ja edut.
\end{quote}

Tässä yhteydessä odotamme siis käsittääkseni kansallista
tietosuojalainsäädäntöä, joka toivottavasti selventäisi arkistoinnin ja
tieteellisen käytön.

\hypertarget{tiedon-kasittely}{%
\chapter{Tiedon käsittely}\label{tiedon-kasittely}}

\hypertarget{anonymisointi}{%
\section{Anonymisointi}\label{anonymisointi}}

\hypertarget{pseudonymisointi}{%
\section{Pseudonymisointi}\label{pseudonymisointi}}

\hypertarget{ida-ymparisto}{%
\section{IDA-ympäristö}\label{ida-ymparisto}}

\bibliography{book.bib,packages.bib}


\end{document}
